\documentclass{article}
\usepackage{hyperref}

\title{A Business Plan to Exploit High
Nutrient Loads in Polluted Waterways for the
Production of Compost and Fertilizer}
\author{Tom Hawkins \\ \texttt{tomahawkins@gmail.com}}

\begin{document}
\maketitle

\begin{abstract}
Agricultural runoff introduces high levels of nutrients
made up of nitrogen and phosphorus
into waterways resulting in large scale algae blooms that 
produce aquatic deadzones and threatens the drinking water
for large populations.
The bloom on western Lake Erie in 2014 poisoned the water supply
for nearly a half a million residence in and around the city of Toledo.
Despite substantial efforts to lower nutrient runoff,
the problem is projected to persist -- Toledo's water
crisis happened even in the face of nearly 30 years of mitigation.  
Though many view waterborne nutrients as waste and sources of pollution,
they are in fact a valuable resource that are undervalued and unexploited.
By growing and harvesting algae on a large scale,
it is possible to extract these nutrients from the water
and convert them into valuable fertilizer for use in
commercial agriculture and small scale farming,
and in doing so prevent the problems associated with high levels of
nutrients in our waterways.
This situation presents a unique "push-pull" business opportunity.
Communities effected by waterborne nutrients are motivated to
push the problem away.  Consumers of fertilizer are motivated
to pull new products to help lower cost and increase crop yield.
The proposed business outlined in this whitepaper fulfills the needs
of both parties and becomes the conduit between the push and pull.
\end{abstract}

\section{Algae Gloom and Doom}

TODO:
Lake Erie, Toledo water crisis, 
unsafe water for swimming and recreation,
e-coli outbreaks (caused from nutrients/algae?)
dead zones in the Gulf of Mexico,
high nitrates in water supplies.

\url{http://www.detroitnews.com/story/news/environment/2015/03/29/lake-erie-algae-relief-years-away/70642990/}


% Farm loby prevents meaningful legislation that would restrict the use of fertilizer.
% Cities around Lake Erie would offer support to ensure security of their water supply.
% This could mean tax incentives, waterway permits, land use permits, etc.

% Currently there is a nearly inexhautable source of nutrients
% that is completely unexploited.

\section{Algae Scrubbing: Growing Algae to Prevent Algae}

Though it may appear counterintutitive,
purposefully growing and harvesting algae is the
most effective way to prevent unwanted algae blooms
in water systems that suffer from high nutrient load.
Specially designed nutrient export systems,
called algae scrubbers\footnote{\url{http://en.wikipedia.org/wiki/Algae_scrubber}},
are now quite common 
in the aquarium industry because they have showing remarkable sucess
at extracting and exporting nutrients from the water column.
An algae scrubber is a device that provides the 
optimal conditions for algal growth: namely light,
water circulation, and an air-water interface.
Under these prime condtions the algae scrubber is able to
grow algae faster than any other part of the system.
By harvesting the algae from the scrubber, nutrients 
are exported and overall levels in the water column decline.
At some point, nutrients become so low that algae can \textit{only}
grow on the scrubber, not in the rest of the system.
For the aquarist, this means a crystal clear display tank
free from unsightly algae growing on the rocks and glass.

Prior to algae scrubbing, nutrients in an aquarium
that accumulated from the break down of food
could only be removed by periodically changing the aquarium water.
But now, algae scrubbers have been so effective that
some aquarists no longer resort to water changes to remove nutrients.
In fact, because the green hair algae that
commonly grow in scrubbers has excellent particulate filtering characteristics,
some aquarists solely rely on algae scrubbers as their primary source of filtration%
\footnote{\url{https://www.youtube.com/watch?v=RAqCZlR\_Un8}}%
\footnote{\url{https://www.youtube.com/watch?v=wD6kA3xDPaM}}
for both nutrients and particulates.

\subsection{Algae Scrubbing on a Larger Scale}

There has been some success at building large scale algae scrubbers%
\footnote{\url{http://www.hydromentia.com/}}%
\footnote{\url{http://www.trl.com/algae/}},
but these demonstration systems are built inland requiring
a complex infrastructure, extensive energy for pumping,
and consume a lot of water to evaporation.

To bypass these problems, we plan to build the
algae scrubber on the very surface of the water 
that needs to be cleaned.


\section{Putting Algae to Use}

It is clear that an enrormous amount of Algae
can be produced by Scrubbing
Lake Erie, the Mississippi River, and many
other bodies of water suffering from high nutrients.
The question is what to do with it?

Though there is a lot of promising research on using algae for
the production of feedstock, fertilizer, and biofuels,
to reduce business risk this plan does not depend on
uncertain future technology.
Instead the business targets the easiest to produce
and most profitable end-product: bags of compost, which you find
at lawn-and-garden and home improvement stores and in the
parking lots of the big box retailers in the early spring and summer.
Our target price point is that of the current price of 
premium potting soil: 2 cubic ft bags sell for \$15.

The core business is:
\begin{itemize}
  \item{Algal production and harvesting.}
  \item{Algal storage and composting.}
  \item{Compost packaging and delivery.}
\end{itemize}

Only once the core business has been established
will new markets be explored, e.g. new harvesting sites,
fertilizer synthesis, biofuels, feedstock, etc.

\section{Enabling Core Technology: The Surface Substrate Array}

Insted of pumping influent through a remote algae scrubber,
our plan is to build the scrubber substrate on the water surface itself.
We call this the Surface Substrate Array (SSA), or simply "The Array".
By colocating the substrate with the water, and relying on
wind and wave action for circulation, this eliminates both the capital
cost and the perpetual energy expense associated with a conventional pumping system.
Further more, floating the substrate on the water surface
has other advantages: being at the surface it captures the maximum solar energy,
it provides the air and water interaction for optimal algal growth,
and it blocks the light below preventing the growth of other
undesirable froms of algae in the water column, e.g. cyanobacteria.

  - Designed to warm the water (black color, deep channels to lower circulation and retain warmth) to lengthen the growing season and increase the growth rate.
  - Square of hexagon titles that are modular, clicks together.
  - Distributed anchor system to maintain position and integrity.
  - Lowered underwater below ice depth in winter.


\section{Overview of Phased Operations}

\subsection{Phase 1: Exploratory Studies}

The goals for phase 1 are to gain better understanding and to optimize
operations for the core business.  This will be done by
developing and testing prototypes of the substrate and array configurations,
harvester systems, and composting systems and processes.
It will also be a time to engage with policy makers,
gain an understand the regulatory requirements of such an operation,
and to begin engaging with the marketplace.
In this phase, the following questions will be answered:

\textbf{Algal production and nutrient extraction:}
\begin{itemize}
  \item{What is expected yield per surface area over the growing season?}
  \item{What is the optimal substrate design to maximize growth rate and length of growing season?}
  \item{How does the array impact wildlife?}
  \item{Should the array be dismantled in winter or can it be stored under the ice?}
  \item{How does nutrient measurements compare around the array with that of surrounding water throughout the growing season?}
  \item{If a large scale algae outbreak occurs, is there any difference in water conditions around the array?}
  \item{Is the added cost of artificial lighting justified for increased growth rate and longer growing season?
        If so, what is the optimal application of artificial lighting?}
\end{itemize}

\textbf{Algal harvesting:}
\begin{itemize}
  \item{How often to harvest for optimal yield?}
  \item{What is the optimal harvester design and process to maximize yield and lower power consumption?}
  \item{How should algae be transported to shore?  Barge or subsurface pipeline?}
  \item{What is the best algae/water separate process to maximize yield and minimize free floating algae?}
  \item{Should the algae/water separation process be performed on the shore or at the array on a boat?}
\end{itemize}

\textbf{Algal composting:}
\begin{itemize}
  \item{How long does it take algae to break down into compost?}
  \item{What is the quality of compost produced?  How well do plants respond?}
  \item{Is it possible to bag the algae at harvest time?}
  \item{How much storage is required to meet requirements of composting and market demand?}
  \item{What are the environmental concerns of a composting facility, e.g. runoff, ventilation, odor?}
  \item{Can environmentally friendly bagging materials be used?}
\end{itemize}

\textbf{Compost marketing:}
\begin{itemize}
  \item{Who are the best customers?}
  \item{What is required to form a contract with a large box store?}
  \item{How is the product inserted in the supply chain?}
  \item{How to effectively create and market branding?}
  \item{At what scale of production can projected costs undercut competitors' prices?}
\end{itemize}


\subsection{Phase 2: Scaling to Initial Compost Production}
% Questions to answer:
%  - Buffer size: How large of a storage area form composte to meet annual needs.
%  - Logistics of yearly operation:
%    - Setting up and taking down array.
%    - Harvesting.
%    - Transportation to composting.
%    - Bagging and prep for sale.
%    - Transportation to distributors.
%  - Approval process from fedral/state acengices.
%  - Apply for incentives.

\subsection{Phase 3: Scaling Up Compost Production, R\&D of Fertilizer Synthesis}

\section{Summary and Conclusion}

\end{document}

