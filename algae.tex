\documentclass{article}
\usepackage{hyperref}
\usepackage{color}

\newcommand{\tom}[1]{{\color{blue}[#1 --Tom]}}

\title{A Business Plan to Exploit High
Nutrient Loads in Polluted Waterways for the
Production of Compost and Fertilizer}
\author{Tom Hawkins \\ \url{https://github.com/tomahawkins/algae}}

\begin{document}
\maketitle

\begin{abstract}
Agricultural runoff introduces high levels of nutrients
consisting of nitrogen and phosphorus
into waterways resulting in large scale algae blooms that 
produce aquatic deadzones and threatens the drinking water for large populations.
The bloom on western Lake Erie in 2014 poisoned the water supply
for nearly a half a million residence in and around the city of Toledo.
Despite substantial efforts to lower nutrient runoff,
the problem is projected to persist -- Toledo's water
crisis happened even in the face of nearly 30 years of phosphorus mitigation.  
Though many view waterborne nutrients as waste and sources of pollution,
they are in fact a valuable resource that are undervalued and unexploited.
By growing and harvesting algae on a large scale,
it is possible to extract these nutrients from the water
and convert them into valuable compost and fertilizer for
gardening and commercial agriculture,
and in doing so alleviate the problems associated
with high levels of nutrients in our waterways.
This situation presents a unique "push-pull" business opportunity.
Communities effected by adverse water conditions are motivated to
push the problem away.  Consumers of fertilizer are motivated
to pull new products to help lower cost and increase crop yield.
This proposed business model fulfills the needs
of both concerns and becomes the conduit between the push and pull.
\end{abstract}

\section{Algae Gloom and Doom}

TODO A background of the problem:
Lake Erie and Toledo water crisis, 
unsafe water for swimming and recreation,
E. coli outbreaks exacerbated by high nutrients,
Mississippi runoff,
dead zones in the Gulf of Mexico,
high nitrates in Iowa water supplies.

\url{http://www.detroitnews.com/story/news/environment/2015/03/29/lake-erie-algae-relief-years-away/70642990/}


\tom{Mention that
farm lobby prevents meaningful legislation that would restrict the use of fertilizer.
Cities around Lake Erie would offer support to ensure security of their water supply,
which could mean tax incentives, waterway permits, land use permits, etc.
}

\section{Algae Scrubbing: Growing Algae to Prevent Algae}

Though it may appear counter intuitive,
purposefully growing and harvesting algae is the
most effective way to prevent unwanted algae blooms
in water systems that suffer from high nutrient load.
Specially designed nutrient export systems,
called algae scrubbers\footnote{\url{http://en.wikipedia.org/wiki/Algae_scrubber}},
are now quite common in the aquarium industry because they have showing remarkable success
at extracting and exporting nutrients from the water column.
An algae scrubber is a device that provides the 
optimal conditions for algal growth: namely light,
water circulation, and an air-water interface.
Under these prime conditions the algae scrubber is able to
grow algae faster than any other part of the system.
By harvesting the algae from the scrubber, nutrients 
are exported and overall nutrient levels in the water column decline.
At some point, nutrients become so low that algae can \textit{only}
grow on the scrubber, not in the rest of the system.
For the aquarist, this means a crystal clear display tank
free from unsightly algae growing on the rocks and glass.

Prior to algae scrubbing, nutrients in an aquarium
that accumulated from the break down of food
could only be removed by periodically changing the aquarium water.
Now however, algae scrubbers have been so effective that
some aquarists no longer resort to water changes.
In fact, because the algae type that commonly grow
in scrubbers has excellent particulate filtering characteristics,
some aquarists solely rely on algae scrubbers as their primary source of filtration%
\footnote{\url{https://www.youtube.com/watch?v=RAqCZlR\_Un8}}%
\footnote{\url{https://www.youtube.com/watch?v=wD6kA3xDPaM}}
for both nutrients and particulates.

\subsection{Algae Scrubbing on a Larger Scale}

There has been some success at building large scale algae scrubbers%
\footnote{\url{http://www.hydromentia.com/}}%
\footnote{\url{http://www.trl.com/algae/}}%
\footnote{\url{http://www.alltech.com/future-of-farming/algae-the-growth-platform}},
but these demonstration systems are built inland requiring
a complex infrastructure, extensive energy for pumping,
and consume a lot of water to evaporation.

To bypass these problems, our plan is to build
scrubbers directly on the surface of the body of water
where the nutrients are to be extracted.

\section{Putting Algae to Use}

Due exceedingly high levels nutrients,
it is clear an enormous amount of Algae can be produced by scrubbing
Lake Erie, the Mississippi River, and other bodies of water.
The question is what to do with all this algae?

Though there is promising research on converting algae into biofuels
and animal feed,
our plan takes a modest, lower risk approach by composting algae into
planting soil and field fertilizer.
One particular product of interest are bags of potting soil
commonly found at home improvement stores and the big box retailers.
The target price point is \$15 for 2 cu-ft.

\section{Analysis of Compost Market}

\tom{How big?  What is the competition?  Garden (bags) vs. farm field fertilizer.}

\section{Business Model}

The core business is focused on the 4 steps of production.  They are:
\begin{enumerate}
  \item{Algal growth}
  \item{Algal harvesting}
  \item{Algal composting}
  \item{Compost packaging}
\end{enumerate}

\subsection{Algal Growth: The Surface Substrate Array}

Instead of pumping influent through a remote algae scrubber,
our plan is to build the scrubber directly on the water surface.
We call this the Surface Substrate Array (SSA), or simply "The Array".
The array is composed of interlocking hexagonal panels that
float on the surface and are anchored in place to the lake or river bottom.
Each panel is composed of a fine mesh to provide the substrate 
for algae to attach and grow.

At the water surface, the array maximizes the collected solar energy
needed for photosynthesis.
It also provides the air and water interface that desirable forms of algae need for rapid growth.
In addition, the panels are designed to maximize heat absorption
to further accelerate growth and lengthen the growing season.
As a side benefit, the array will block light below it
preventing growth of undesirable free-floating forms of algae,
such as cyanobacteria, which cause problems for beaches and water treatment plants.

By colocating the substrate with the water, and relying on
wind and wave action for circulation, this eliminates the capital
costs and the perpetual energy expense associated with a pumping system
needed for an inland facility.


\tom{Other things to mention:
Distributed anchor system to maintain position and integrity.
Lowered underwater below ice depth in winter.
}

\subsection{Algal Harvesting}

Algal harvester: a robot that crawls the array
with a hedgetrimmer and pumps the algae clippings
back to the collection site (a boat/barge or on shore).

\subsection{Algal Composting}
\subsection{Compost packaging}

\section{Overview of Phased Operations}
\subsection{Phase 1: Exploratory Studies}

The goals for phase 1 are to gain better understanding and to optimize
operations for the core business.  This will be done by
developing and testing prototypes of the array,
harvester systems, and composting systems and processes.
It will also be a time to engage with policy makers,
gain an understand the regulatory requirements of such an operation,
and to begin engaging with the marketplace.
In this phase, the following questions will be answered:

\textbf{Algal growth and nutrient extraction:}
\begin{itemize}
  \item{What is expected yield per surface area over the growing season?}
  \item{What is the optimal substrate design to maximize growth rate and length of growing season?}
  \item{How does the array impact wildlife?}
  \item{Should the array be dismantled in winter or can it be stored under the ice?}
  \item{How does nutrient measurements compare around the array with that of surrounding water throughout the growing season?}
  \item{If a large scale algae outbreak occurs, is there any difference in water conditions around the array?}
  \item{Is the added cost of artificial lighting justified for increased growth rate and longer growing season?
        If so, what is the optimal application of artificial lighting?}
\end{itemize}

\textbf{Algal harvesting:}
\begin{itemize}
  \item{How often to harvest for optimal yield?}
  \item{What is the optimal harvester design and process to maximize yield and lower power consumption?}
  \item{How should algae be transported to shore?  Barge or subsurface pipeline?}
  \item{What is the best algae/water separate process to maximize yield and minimize free floating algae?}
  \item{Should the algae/water separation process be performed on the shore or at the array on a boat?}
\end{itemize}

\textbf{Algal composting:}
\begin{itemize}
  \item{How long does it take algae to break down into compost?}
  \item{What is the quality of compost produced?  How well do plants respond?}
  \item{Is it possible to bag the algae at harvest time?}
  \item{How much storage is required to meet requirements of composting and market demand?}
  \item{What are the environmental concerns of a composting facility, e.g. runoff, ventilation, odor?}
  \item{Can environmentally friendly bagging materials be used?}
\end{itemize}

\textbf{Compost marketing:}
\begin{itemize}
  \item{Who are the best customers?}
  \item{What is required to form a contract with a large box store?}
  \item{How is the product inserted in the supply chain?}
  \item{How to effectively create and market branding?}
  \item{At what scale of production can projected costs undercut competitors' prices?}
\end{itemize}


\subsection{Phase 2: Scaling to Initial Production}

\subsection{Phase 3: Scaling Up Production, Expanding to New Sites}

\section{Cost and Profit Predictions}

\section{Summary and Conclusion}

Hooray for slimy green stuff!

\end{document}

